\documentclass{article}
\usepackage[utf8]{inputenc}

\title{Muon lifetime data system calibration and capture electronics configuration}
%\title{Muon Lifetime Data Acquisition Configuration}

\author{Vishnu Chavva and Elliot Urriola }
\date{February 6, 2019}

\begin{document}

\maketitle

%Abstract section
\begin{center}
    
    \section*{Abstract}
    Hello world.  
\end{center}

%Introduction section
\section*{Introduction}

\hspace{3.5mm} Supernovae located well outside the Earth's solar system often emit high-energy radiation called cosmic rays. These cosmic rays are mostly comprised of protons [1]. %[cosmic ray muons slides]
Eventually, the cosmic rays reach Earth's  upper atmosphere, where the constituent protons most frequently collide with diatomic molecules [1]. %[cosmic ray muons slides]
This interaction begins a series of decay processes which produce an array of new particles. Among these product particles are muons. Muons are charged, fundamental fermions that are related to electrons. Cosmic rays serve as the most abundant source of muons on earth, providing approximately 1 muon per square centimeter every minuite at sea level.   

Current areas of research involving the muon often require complicated collider systems and sophisticated electronics. Addtionally, progressing this technology has become an important part of physics research beyond the standard model [2]. %[collider]
However, much simpler experiments have been designed to exemplify the research concepts used in high energy physics [3]. %[undergrad]

%These simplified experiments focus on a determination of the muon lifetime. The readout electronics are much simpler than the electronics built for collider experiments, and often use scintillators attached to photomultiplier tubes (PMT) as a detection mechanism [0]. %[undergrad] 


Specifically, simplified experiments focus on a determination of muon lifetime in order to investigate weak interaction phenomena. Muon decay is a process driven entirely by the weak force [4]. %[gfriffiths pcl physics]
So, it is possible to measure the strength of the weak force by measuring muon lifetime. This relationship is summarized by Equation 1 where $\tau_{\mu}$ is muon lifetime and \textit{G$_{F}$} is a constant called the Fermi Coupling Constant. A rigorous derivation of Equation 1 is found in Griffiths chapter 9.2 [4]. %[Griffiths]   

%Equation 1
\begin{center}
    
    $\tau_{\mu} = \frac{192 \pi^3}{\textit{G$_{F}$}^2 m_{\mu}^5}$ \hspace{10mm} \textbf{(1)}
    
    \vspace{5mm}
    
    \textbf{Equation 1} Muon lifetime is proportional to the inverse square of the Fermi Coupling Constant.
    
\end{center}


\textit{G$_{F}$} is the quantity that describes the strength of the weak force. As such, \textit{G$_{F}$} is necessary for all calculations of weak interaction phenomena [5]. %[3 constants] 
Thus, a precise value for \textit{G$_{F}$} improves the accuracy of such calculations.


From Equation 1, it is clear that determining \textit{G$_{F}$} could be achieved by a measurement of muon lifetime. The lifetime of a muon can be thought of as the inverse of the muon disappearance rate $\Gamma_{diss.}$. This relationship is shown in Equation 2. 

%Equation 2
\begin{center}

    %dissapearance rate and muon lifetime relationship
    %$\tau_{\mu} = \frac{1}{\Gamma_{diss.}}$ \hspace{10mm} \textbf{(2)}
    $\Gamma_{diss.} = \frac{1}{\tau_{\mu}}$ \hspace{10mm} \textbf{(2)}
    
    \vspace{5mm}
    
    \textbf{Equation 2} Muon lifetime is the inverse of disappearance rate
    
\end{center}

A muon can dissappear by decaying according to the process shown in Equation 3. These decay processes are different depending on the charge of the muon at the time of decay. 

%Equation 3
\begin{center}
    
    %positive muon decay
    $\mu^{+} \rightarrow e^{+} + \nu_{e} + \overline{\nu}_{\mu}$ \hspace{10mm} \textbf{(3.a)}
    
    \vspace{1 mm}
    
    %negative muon decay; symmetric
    $\mu^{-} \rightarrow e^{-} + \overline{\nu}_{e} + \nu_{\mu}$ \hspace{10mm} \textbf{(3.b)}
    
    \vspace{5mm}
    
    \textbf{Equation 3} Muon decay equations for \textbf{a)} positively charged muons. \textbf{b)} negatively charged muons 
    
\end{center}
  
However, negatively charged muons do not always decay. Some may interact with protons from the incoming cosmic rays and disappear according to Equation 4. 

%Equation 4
\begin{center}
    
    %negative muon decay; symmetric
    $\mu^{-} + \textit{p} \rightarrow \textit{n} +  \nu_{\mu}$ \hspace{10mm} \textbf{(4)}
    
    \vspace{5mm}
    
    \textbf{Equation 4} Alternative disappearance process for negatively charged muons 
    
\end{center}

Thus, an expansion of Equation 2 holds for negatively charged muons. The expansion is seen in Equation 5. 
%Equation 5
\begin{center}
    
    %expansion of muon disappearance rate
    $\Gamma_{diss._{\mu^{-}}} = \Gamma_{decay} + \Gamma_{interact}$ \hspace{10mm} \textbf{(5)}
    
    \vspace{5mm}
    
    \textbf{Equation 5} Expansion of Equation 2 for negatively charged muons
    
\end{center}

The additional disappearnce mode for negatively charged muons further complicates the muon lifetime measurement. That is, while the detection electronics in these simplified experiments are sensitive to the energy deposited by the muon decay products (Equation 3), they are unable to discern exactly how the muon disappeared. Adjustments are made when calculating the muon lifetime based on data acquired during the experiment, and prior knowledge of cosmic ray composition [1]. %[cosmic ray composition] 

The main goal of this experiment is to calculate \textit{G$_{F}$} from a measurement of muon lifetime. We follow a procedure similar to the simplified procedure [3] %undergrad
described above to record muon decay events over several days. The electronics configuration and data acquisition will be used in further muon investigation experiments. 
 

 










 






 
 
\section*{References}

\hspace{4mm} [1] \hspace{1mm}  Anthony Tatum, "Cosmic Ray Muons," \textbf{https://uncw.edu/phy}\newline (February 1, 2019)

\vspace{3mm}

[2] \hspace{1mm} M. S. Zisman, Lawrence Berkeley National Laboratory (2011)

\vspace{3 mm}



[3] \hspace{1mm}  D. Bosnar, et. al, A simple setup for cosmic muon lifetime measurements, \textit{Eur. J. Phys.} \textbf{39}, 4 (2018)
\vspace{3 mm}

[4] \hspace{1mm} D. Griffiths, "Decay of the Muon," in \textit{Introduction to Elementary Particles}, Wiley-VCH; 2nd edition, Ch. 9, pp. 310 - 315

\vspace{3 mm}

[5] \hspace{1mm} T. van Ritbergen and R. G. Stuart, Complete 2-loop quantum electrodynamic contributions to the muon lifetime in the fermi model, \textit{Phys. Rev. Lett.} \textbf{82}, 488 (1999).























\end{document}
